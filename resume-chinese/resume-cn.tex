\documentclass[10pt, a4paper, sans]{moderncv}

%% ModernCV themes
\moderncvstyle{banking}
\moderncvcolor{blue}
\renewcommand{\familydefault}{\sfdefault}
\nopagenumbers{}

%% Character encoding
\usepackage[utf8]{inputenc}
\usepackage[UTF8]{ctex}

%% Adjust the page margins
\usepackage[scale=0.75]{geometry}
\geometry{top=2.0cm, bottom=2.0cm}

\usepackage{fancyhdr}
\pagestyle{fancy}
% \lhead{\textbf{Program: Computer Science and Engineering Ph.D. \\ U-M ID: 98197295}}

%% Personal data
\firstname{张}
\familyname{震元}
\title{简历}
%\address{Room 1601, No.12, 665 Dongchangzhi Road}{Shanghai, 200080}
\address{5C, 1811 Willowtree Ln, Ann Arbor}{Michigan, 48105}
\mobile{+1~(734)~882~3816}
\email{cryscan@umich.edu}
\homepage{cryscan.github.io/profile}

%%------------------------------------------------------------------------------
%% Content
%%------------------------------------------------------------------------------
\begin{document}
\makecvtitle

\section{教育背景}
%\cventry{year--year}{Degree}{Institution}{City}{ \textit{Grade} }{Description}  % arguments 3 to 6 can be left empty
\cventry{2021年9月 -- 2022年12月}
{计算机科学硕士}
{密歇根大学安娜堡分校}
{安娜堡}
{}
{\cvitem{核心课程}{并行计算、运动机器人、范畴论}}

\cventry{2019年9月 -- 2021年5月}
{计算机科学学士}
{密歇根大学安娜堡分校}
{安娜堡}
{GPA 3.8 out of 4.0}
{\cvitem{核心课程}{计算机体系结构、数据结构与算法、操作系统、编译原理、游戏开发、机器人运动学和动力学}}

\cventry{2017年9月 -- 2021年8月}
{电子与计算机工程学士}
{上海交通大学}
{上海}
{GPA 3.6 out of 4.0}
{\cvitem{核心课程}{工程随机方法、微分方程、线性代数、离散数学}}

\section{项目经历}
\cventry{2022年1月 -- 2022年2月}
{程序}
{体素光锥跟踪实时全局光照插件}{}{}
{
	为开源游戏引擎\texttt{Bevy}开发的全局光照扩展插件
	\begin{itemize}
		\item 使用现代GPU图形和计算API: WebGPU实现
		\item 编写了体素化场景、生成Mip-map和光锥跟踪着色的管线
		\item 发布到\texttt{crates.io}供其他开发者使用
	\end{itemize}
}

\cventry{2021年11月 -- 2022年1月}
{设计、程序}
{大规模GPU并行$A^\star$搜索}{}{}
{
	一个科研项目,在GPU上实现大规模并行$A^\star$搜索
	\begin{itemize}
		\item 使用\texttt{CUDA}实现GPU上的堆、哈希表和内存池
		\item 在Quadro RTX 4000上取得了和单核i7 8700版本超过10倍的加速
	\end{itemize}
}
\cventry{2021年1月 -- 2021年4月}
{个人项目}
{计算机生成动画}
{}{}
{密歇根大学自主研究项目,将机器人学中的动作生成运用于角色动画中
	\begin{itemize}
		\item 实现了一款碰撞可求导的物理模拟程序,以用来离线生成真实的角色动画片段
		\item 实现了\texttt{C++}到\texttt{C\#}的接口,可在\texttt{Unity}引擎中调用动画生成器
		\item 使用轨迹规划方法预先生成包含角色IK控制点动画片段的数据库
		\item 使用动作匹配 (Motion Matching) 根据用户输入实时合成动画,并使用IK补全其全身动画
	\end{itemize}
}

% \section{Activities and Honors}
% \cventry{Mar. 2018 -- Apr. 2018}
% {Team Leader}
% {The 10th SJTU Mechanical Innovation Competition for Freshmen}
% {}{Entered the second round}
% {A college-level competition in robot designing}

\section{计算机技术}
\cvitem{编程语言}{
	\begin{itemize}
		\item \cvitem{\texttt{C++}}{最为熟悉的语言,有良好的编码风格;有多个项目经验}
		\item \cvitem{\texttt{Rust}}{理解所有权,生命周期等概念;有项目经验}
	\end{itemize}}
\cvitem{游戏引擎}{\texttt{Unity}, \texttt{Bevy} (由\texttt{Rust}编写)}
\cvitem{其他}{\texttt{Git} (版本控制), \texttt{Jira} (项目管理), \texttt{Blender} (3D建模)}

\end{document}