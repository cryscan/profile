\documentclass[10pt, a4paper, sans]{moderncv}

%% ModernCV themes
\moderncvstyle{banking}
\moderncvcolor{blue}
\renewcommand{\familydefault}{\sfdefault}
\nopagenumbers{}

%% Character encoding
\usepackage[utf8]{inputenc}
\usepackage[UTF8]{ctex}

%% Adjust the page margins
\usepackage[scale=0.75]{geometry}
\geometry{top=2.0cm, bottom=2.0cm}

\usepackage{fancyhdr}
\pagestyle{fancy}
% \lhead{\textbf{Program: Computer Science and Engineering Ph.D. \\ U-M ID: 98197295}}

%% Personal data
\firstname{张}
\familyname{震元}
\title{简历}
%\address{Room 1601, No.12, 665 Dongchangzhi Road}{Shanghai, 200080}
\address{5C, 1811 Willowtree Ln, Ann Arbor}{Michigan, 48105}
\mobile{+1~(734)~882~3816}
\email{cryscan@umich.edu}

%%------------------------------------------------------------------------------
%% Content
%%------------------------------------------------------------------------------
\begin{document}
\makecvtitle

\section{教育背景}
%\cventry{year--year}{Degree}{Institution}{City}{ \textit{Grade} }{Description}  % arguments 3 to 6 can be left empty
\cventry{2019年9月 -- 2021年5月}
{计算机科学学士}
{密歇根大学安娜堡分校}
{安娜堡}
{GPA 3.9 out of 4.0}
{\cvitem{核心课程}{计算机体系结构、数据结构与算法、操作系统、编译原理、游戏开发}}

\cventry{2017年9月 -- 2021年8月}
{电子与计算机工程学士}
{上海交通大学}
{上海}
{GPA 3.6 out of 4.0}
{\cvitem{核心课程}{工程随机方法、微分方程、线性代数、离散数学}}

\section{个人经历}
\cventry{2021年1月 -- 至今}
{个人项目}
{计算机生成动画}
{}{}
{密歇根大学自主研究项目,将机器人学中的动作生成运用于角色动画中
	\begin{itemize}
		\item 为轨迹规划算法软件 \texttt{towr} 实现了 \texttt{C++} 接口
		\item 使用轨迹规划方法生成包含角色控制点动画片段的数据库
		\item 将生成的控制点动作根据用户输入实时绑定在游戏角色上,并补全其全身动画
	\end{itemize}
}

\cventry{2020年10月 -- 2020年12月}
{设计、程序}
{游戏 《The Inside Man》}
{}{}
{密歇根大学游戏开发课程项目
	\begin{itemize}
		\item 设计了一套基于“规划-执行”的核心玩法
		\item 设计了项目的代码架构
		\item 使用 \texttt{Goal Oriented Action Planning} 实现AI的复杂逻辑
	\end{itemize}
}

\cventry{2020年5月 -- 至今}
{研究助理}
{Trap Aware Model Predictive Control}
{}{}
{一种在线的、基于模型的控制算法,能在其未曾见过的环境中逃脱
	\begin{itemize}
		\item 实现了基于 \texttt{Guided Policy Search} 的 baseline
		\item 实现了基于 \texttt{Soft Actor-Critic} 的 baseline
	\end{itemize}
}

\cventry{2020年6月 -- 2020年8月}
{程序}
{WolverineSoft Studio 社团独立游戏 《Desolation Place》}
{}{}
{一款第一人称恐怖潜行游戏
	\begin{itemize}
		\item 由社团30人合作完成
		\item 使用轨迹优化方法生成敌人爬行动画
		\item 实现所有玩家与环境物品交互
		\item 实现叙事系统
	\end{itemize}
}

\cventry{2018年11月 -- 2019年4月}{学生}{本科生科研项目}{}{}
{{评估深度强化学习算法}
	\begin{itemize}
		\item 搭建了一个统一的环境来比较各种强化学习算法
		\item 设计并实现了一套自动化的评估、比较流程
	\end{itemize}
}

% \section{Activities and Honors}
% \cventry{Mar. 2018 -- Apr. 2018}
% {Team Leader}
% {The 10th SJTU Mechanical Innovation Competition for Freshmen}
% {}{Entered the second round}
% {A college-level competition in robot designing}

\section{计算机技术}
\cvitem{编程语言}{
	\begin{itemize}
		\item \cvitem{\texttt{C++}}{最为熟悉的语言,有良好的编码风格;有多个项目的使用经验}
		\item \cvitem{\texttt{C\#}}{在 \texttt{Unity} 引擎中使用;有多个项目的使用经验}
		\item \cvitem{\texttt{Rust}}{理解所有权,生命周期等概念;有项目经验}
	\end{itemize}}
\cvitem{游戏引擎}{\texttt{Unity}, \texttt{Amethyst}}
% \cvitem{Assets Creating}{\texttt{Blender} for 3D modeling and animating}

\end{document}